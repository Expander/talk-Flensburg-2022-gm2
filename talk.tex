\documentclass[12pt]{beamer}

\usecolortheme[dark,accent=yellow]{solarized}
\beamertemplatetransparentcovered
\setbeamertemplate{navigation symbols}{} % remove navigation symbols
% \setbeamertemplate{footline}[page number]
\setbeamertemplate{footline}{
  \hfill%
  \usebeamercolor[fg]{page number in head/foot}%
  \usebeamerfont{page number in head/foot}%
  \setbeamertemplate{page number in head/foot}[framenumber]%
  \usebeamertemplate*{page number in head/foot}\kern1em\vskip2pt%
}
\setbeamerfont{page number in head/foot}{size=\small}

\usepackage[german]{babel}
\usepackage{times}
\usepackage{esvect}\renewcommand{\vec}[1]{\vv{#1}}
\usepackage{setspace}\setstretch{1.5}

\title{The magic is always in the details}
\subtitle{The search for new physics with the muon}

\author[Voigt]{Alexander Voigt}
\institute[HS Flensburg]{Hochschule Flensburg}
\date{Planetarium Talks 2022}

\begin{document}

%%%%%%%%%%%%%%%%%%%%%%%%%%%%%%%%%%%%%%%%%%%%%%%%%%

\begin{frame}
  \titlepage
\end{frame}

%%%%%%%%%%%%%%%%%%%%%%%%%%%%%%%%%%%%%%%%%%%%%%%%%%

\begin{frame}{Table of Contents}
  \tableofcontents
\end{frame}

%%%%%%%%%%%%%%%%%%%%%%%%%%%%%%%%%%%%%%%%%%%%%%%%%%

\section{What is the world made of?}

\begin{frame}{\insertsection}
  Vom Kristall zum Quark
\end{frame}

%%%%%%%%%%%%%%%%%%%%%%%%%%%%%%%%%%%%%%%%%%%%%%%%%%

\section{The Standard Model of Particle Physics}

\begin{frame}{\insertsection}
  SM-Grafik
\end{frame}

%%%%%%%%%%%%%%%%%%%%%%%%%%%%%%%%%%%%%%%%%%%%%%%%%%

\section{Open questions}

\begin{frame}{\insertsection}
  \begin{itemize}
  \item Dark Matter
  \item Unification of Forces
  \item Hierarchy Problem
  \end{itemize}
\end{frame}

%%%%%%%%%%%%%%%%%%%%%%%%%%%%%%%%%%%%%%%%%%%%%%%%%%

\section{What is a magnetic moment?}

\begin{frame}{\insertsection}
  Ursache des Magnetismus bei einem Permanentmagneten:
  \begin{itemize}
  \item Bahndrehimpuls
  \item Spin
  \end{itemize}
  der Atome
\end{frame}

\begin{frame}{\insertsection}
  Magnetisches Moment $\vec{m}$:
  \begin{itemize}
  \item Beschreibt WW eines magnetischen Dipols mit einem Magnetfeld
  \item Betrag: "`Stärke der WW"'
  \item Richtung: Angestrebte Ausrichtung zum Magnetfeld
  \end{itemize}
\end{frame}

%%%%%%%%%%%%%%%%%%%%%%%%%%%%%%%%%%%%%%%%%%%%%%%%%%

\section{History of the measurement}

\begin{frame}{\insertsection}
\end{frame}

%%%%%%%%%%%%%%%%%%%%%%%%%%%%%%%%%%%%%%%%%%%%%%%%%%

\section{The Anomalous Magnetic Moment}

\begin{frame}{\insertsection}
  Relative deviation of $g$ from the value 2:
  \begin{align*}
    a := \frac{g-2}{2}
  \end{align*}
\end{frame}

%%%%%%%%%%%%%%%%%%%%%%%%%%%%%%%%%%%%%%%%%%%%%%%%%%

\section{How can we explain the deviation?}

\begin{frame}{\insertsection}
  Möglicher Grund für Abweichung: Es gibt noch weitere unentdeckte
  Elementarteilchen!
  \begin{itemize}
  \item zusätzliche Higgs-Teilchen?
  \item zusätzliche Quarks oder Leptonen?
  \item Supersymmetrie?
  \end{itemize}
\end{frame}

\end{document}
