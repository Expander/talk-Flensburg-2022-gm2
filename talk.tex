\documentclass[12pt,notes]{beamer}

\usecolortheme[dark,accent=yellow]{solarized}
\setbeamercovered{transparent=0}
\setbeamertemplate{navigation symbols}{} % remove navigation symbols
% \setbeamertemplate{footline}[page number]
\setbeamertemplate{footline}{
  \hfill%
  \usebeamercolor[fg]{page number in head/foot}%
  \usebeamerfont{page number in head/foot}%
  \setbeamertemplate{page number in head/foot}[framenumber]%
  \usebeamertemplate*{page number in head/foot}\kern1em\vskip2pt%
}
\setbeamerfont{page number in head/foot}{size=\small}
\setbeamerfont{note page}{size=\scriptsize}
\addtobeamertemplate{note page}{\setbeamerfont{itemize/enumerate subbody}{size=\tiny}}{}
\setbeamertemplate{bibliography item}{\insertbiblabel}

\usepackage[german]{babel}
\usepackage{times}
\usepackage{esvect}\renewcommand{\vec}[1]{\vv{#1}}
\usepackage{setspace}\setstretch{1.5}
\usepackage{tikz}
\usepackage{pgfplots}\pgfplotsset{compat=1.17}
\usepackage{siunitx}
\usepackage[backend=biber,natbib=true,style=numeric,sorting=none]{biblatex}
\usepackage{multimedia}

\addbibresource{talk.bib}

%%%%%%%%%%%%%%%%%%%%%%%%%%%%%%%%%%%%%%%%%%%%%%%%%%

\usetikzlibrary{calc,decorations.markings,decorations.pathmorphing,positioning,shapes}

% default arrow style
\tikzset{>=latex}

\tikzset{photon/.style={decorate, decoration={snake, segment length=3mm, amplitude=0.8mm}}}

\tikzset{%
    ->-/.style={postaction={decorate},decoration={%
            markings,mark=at position #1 with {\arrow{>}}%
        }%
    },%
    ->-/.default=.5,%
    -<-/.style={postaction={decorate},decoration={%
            markings,mark=at position #1 with {\arrowreversed{>}}%
        }%
    },%
    -<-/.default=.5%
}

% arrows on the field lines
\tikzstyle directed=[postaction={decorate,decoration={markings,
  mark=at position .2 with {\arrowreversed[scale=1.5]{>}},
  mark=at position .8 with {\arrowreversed[scale=1.5]{>}}}}]

% field lines
\tikzstyle fLines=[thick,directed]

\newcommand{\magnet}[3]{%
    \def\lmag{#1}  % length of magnet
    \def\wmag{#2}  % thickness of magnet
    \def\nc{#3}    % no. of lines = 2*\nc+1
    \begin{scope}
      \coordinate (A) at (-\lmag/2,\wmag/2);
      \coordinate (B) at (\lmag/2,-\wmag/2);
      \draw[fill, color=green](A) rectangle ++(\lmag/2,-\wmag) node[black,midway]{S};
      \draw[fill, color=red](0,-\wmag/2) rectangle ++(\lmag/2,\wmag) node[black,midway]{N};
      \clip (-5,-3) rectangle (5,3);
      \foreach \r in {1,...,\nc}{
        \draw[fLines]($(A)-(0,0.5*\r*\wmag/\nc)$) arc(({270-asin(\lmag/(2*\r))}):({-90+asin(\lmag/(2*\r))}):\r);
        \draw[fLines]($(B)+(0,0.5*\r*\wmag/\nc)$) arc(({90-asin(\lmag/(2*\r))}):({-270+asin(\lmag/(2*\r))}):\r); }
      \draw[fLines] (-\lmag/2,0) -- ++(-6,0);
      \draw[fLines] (\lmag/2,0) ++(6,0)--(\lmag/2,0);
    \end{scope}
    \draw[blue,->,line width=3] (-\nc/8,2) -- ++(\nc/4,0) node[midway,above] {$\vec{m}$};
}

% Quantum correction, #1 = coordinate for origin, #2 = angle
\newcommand{\QC}[2]{%
  \begin{scope}[shift={#1},rotate around={#2:(0,0)}]
    % \draw[blue,fill] (0,0) circle (0.05);
    % \draw[blue,fill] (1,0) circle (0.05);
    \pgfmathsetmacro{\angl}{45}
    \draw[->-=0.7,blue] (0,0) to[out={-\angl},in={180+\angl}] node[pos=0.3,green] (a) {} (1,0);
    \draw[->-=0.7,blue] (0,0) to[out={\angl}, in={90+\angl}]  node[pos=0.3,green] (b) {} (1,0);
    \draw[green,fill] (a) circle (0.05);
    \draw[green] (b) circle (0.05);
  \end{scope}
}

\newcommand{\SMtable}{%
    \begin{tikzpicture}[node distance = 2.5em, auto]
      \node[quark] (u) {$u$};
      \node[quark, below of=u] (d) {$d$};
      \node[quark, right of=u] (c) {$c$};
      \node[quark, below of=c] (s) {$s$};
      \node[quark, right of=c] (t) {$t$};
      \node[quark, below of=t] (b) {$b$};
      \node[lepton, below of=d] (ne) {$\nu_e$};
      \node[lepton, below of=ne] (e) {$e$};
      \node[lepton, right of=ne] (nm) {$\nu_\mu$};
      \node[lepton, below of=nm] (m) {$\mu$};
      \node[lepton, right of=nm] (nt) {$\nu_\tau$};
      \node[lepton, below of=nt] (ta) {$\tau$};
      \node[gauge, right of=t] (gamma) {$\gamma$};
      \node[gauge, below of=gamma] (g) {$g$};
      \node[gauge, below of=g] (Z) {$Z$};
      \node[gauge, below of=Z] (W) {$W$};
      \node[scalar, right of=W] (H) {$h$};
      \node[rotate=90] (quarks)  at ($(u)!0.5!(d)+(-1,0)$)  {quarks};
      \node[rotate=90] (leptons) at ($(ne)!0.5!(e)+(-1,0)$) {leptons};
      \node[below of=H] (higgs) {Higgs};
      \node[above of=gamma, align=center] (gauge) {gauge\\[-0.5em] bosons};
    \end{tikzpicture}
}

%%%%%%%%%%%%%%%%%%%%%%%%%%%%%%%%%%%%%%%%%%%%%%%%%%

\tikzstyle{block} = [rectangle, draw, text width=7em, text centered, minimum height=2em]

\tikzstyle{quark}     = [rectangle, black, draw, fill=yellow, minimum width=2em, text centered, minimum height=2em]
\tikzstyle{lepton}    = [rectangle, black, draw, fill=red!50, minimum width=2em, text centered, minimum height=2em]
\tikzstyle{gauge}     = [circle   , black, draw, fill=green , minimum size=2em, inner sep=0pt, text centered]
\tikzstyle{scalar}    = [diamond  , black, draw, fill=blue!40, minimum width=2.3em, text centered, minimum height=2.3em, inner sep=0pt]
\tikzstyle{goldstone} = [diamond  , black, draw, dashed, fill=blue!30, minimum width=2.3em, text centered, minimum height=2.3em, inner sep=0pt]
\tikzstyle{squark}    = [diamond  , black, draw, fill=yellow, minimum width=2.3em, text centered, minimum height=2.3em, inner sep=0pt]
\tikzstyle{slepton}   = [diamond  , black, draw, fill=red!50, minimum width=2.3em, text centered, minimum height=2.3em, inner sep=0pt]
\tikzstyle{gaugino}   = [rectangle, black, draw, fill=green , minimum size=2em, inner sep=0pt, text centered]
\tikzstyle{higgsino}  = [rectangle, black, draw, fill=blue!40  , minimum width=2em, text centered, minimum height=2em]
\tikzstyle{inert}     = [diamond  , black, draw, fill=teal!80, minimum width=2.3em, text centered, minimum height=2.3em, inner sep=0pt]
\tikzstyle{inertino}  = [rectangle, black, draw, fill=teal!80, minimum width=2em, text centered, minimum height=2em]
\tikzstyle{phantom}   = [rectangle, black, minimum width=2em, text centered, minimum height=2em]

%%%%%%%%%%%%%%%%%%%%%%%%%%%%%%%%%%%%%%%%%%%%%%%%%%

\newcommand{\Exp}{\text{Exp}}
\newcommand{\SM}{\text{SM}}

% CERN 1979 \cite{CERN-Mainz-Daresbury:1978ccd}
% aμ+ = 1165911(11) × 10−9
% aμ− = 1165937(12) × 10−9
%
% Combine the two measurements:
% https://stats.stackexchange.com/questions/193987/how-to-combine-two-measurements-of-the-same-quantity-with-different-confidences
% a1 = 1165911e-9
% d1 = 11e-9
% a2 = 1165937e-9
% d2 = 12e-9
% comb(a1, d1, a2, d2) = (a1/d1^2 + a2/d2^2) / (1/d1^2 + 1/d2^2)
% dcomb(d1, d2) = sqrt(1 / (1/d1^2 + 1/d2^2))

\newcommand{\amuCERNMuP}{11659110} % e-10
\newcommand{\numamuCERNMuP}{\num{\amuCERNMuP}}
\newcommand{\DamuCERNMuP}{110}

\newcommand{\amuCERNMuM}{11659370} % e-10
\newcommand{\numamuCERNMuM}{\num{\amuCERNMuM}}
\newcommand{\DamuCERNMuM}{120}

\newcommand{\amuCERN}{11659229} % e-10
\newcommand{\numamuCERN}{\num{\amuCERN}}
\newcommand{\DamuCERN}{81} % e-10

% BNL-E821
\newcommand{\amuBNL}{11659208.0} % e-10
\newcommand{\numamuBNL}{\num{\amuBNL}}
\newcommand{\DamuBNL}{6.3}

% FNAL 2021
% https://journals.aps.org/prl/abstract/10.1103/PhysRevLett.126.141801
\newcommand{\amuFNAL}{11659204.0} % e-10
\newcommand{\DamuFNAL}{5.4}
\newcommand{\numamuFNAL}{\num{\amuFNAL}}

% experimental, combined
% https://journals.aps.org/prl/abstract/10.1103/PhysRevLett.126.141801
\newcommand{\amuExp}{11659206.1}
\newcommand{\numamuExp}{\num{\amuExp}}
\newcommand{\DamuExp}{4.1} % uncertainty
\newcommand{\aeExp}{11596521807.3}
\newcommand{\numaeExp}{\num{\aeExp}}
\newcommand{\DaeExp}{2.8}

% SM
\newcommand{\amuSM}{11659181.0}
\newcommand{\numamuSM}{\num{\amuSM}}
\newcommand{\DamuSM}{4.3}
\newcommand{\aeSM}{11596521816.4}
\newcommand{\numaeSM}{\num{\aeSM}}
\newcommand{\DaeSM}{7.7}

\definecolor{red}{rgb}{1.0,0.2,0.2}
\definecolor{blue}{rgb}{0,0.7,1.0}
\definecolor{green}{rgb}{0,1.0,0.5}

%%%%%%%%%%%%%%%%%%%%%%%%%%%%%%%%%%%%%%%%%%%%%%%%%%

\title{The magic is always in the details}
\subtitle{The search for new physics with the muon}

\author[Voigt]{Alexander Voigt}
\institute[HS Flensburg]{Hochschule Flensburg}
\date{Planetarium Talks 2022}

\begin{document}

%%%%%%%%%%%%%%%%%%%%%%%%%%%%%%%%%%%%%%%%%%%%%%%%%%

\begin{frame}
  \titlepage
\end{frame}

%%%%%%%%%%%%%%%%%%%%%%%%%%%%%%%%%%%%%%%%%%%%%%%%%%

% \begin{frame}{Table of Contents}
%   \tableofcontents
% \end{frame}

%%%%%%%%%%%%%%%%%%%%%%%%%%%%%%%%%%%%%%%%%%%%%%%%%%

\section{Introduction}

\begin{frame}{}
  \begin{center}
    \includegraphics[width=0.8\textwidth]{img/Gravitationell-lins-Hubble}
    \cite{abell}
  \end{center}
\end{frame}

\note{
  \begin{itemize}
  \item Start with picture of the galaxy cluster ``Abell 1689'' from the Hubble telescope
  \item look closely: you'll see a gravity lens effect
  \item So, there must be some massive object(s) between the galaxy
    cluster and us.
  \item If one counts the number of visible objects (stars), one finds
    that (assuming ART is correct), the cumulative mass of the stars
    is not enough to explain this gravity lens effect.
  \item So, there must be some invisible massive matter between the
    galaxy cluster and us. This is what astronomers and cosmologists
    call ``Dark Matter''.
  \end{itemize}
}

%%%%%%%%%%%%%%%%%%%%%%%%%%%%%%%%%%%%%%%%%%%%%%%%%%

\begin{frame}{}
  \begin{center}
    \includegraphics[width=0.8\textwidth]{img/Galaxy-M101}
    \cite{M101}
  \end{center}
\end{frame}

\note{
  \begin{itemize}
  \item Look at the photo from the Hubble telescope, it shows the
    spiral galaxy M101, which is rotating.
  \item If one counts the number of (visible) stars, one can estimate
    the mass distribution of the galaxy.
  \item From the mass distribution, one can predict (using ART) the
    speed at which the outer stars move around the center.
  \item However, one finds that the speed tends to be significantly
    smaller than expected.
  \item This effect could be explained if there would be some
    invisible (Dark Matter) that fills the galaxy.
  \end{itemize}
}

%%%%%%%%%%%%%%%%%%%%%%%%%%%%%%%%%%%%%%%%%%%%%%%%%%

\begin{frame}{}
  \begin{center}
    \movie[width=\textwidth,autostart,loop]{\includegraphics[width=\textwidth]{videos/dark_matter_affecting_galaxy_rotation.png}}{videos/dark_matter_affecting_galaxy_rotation.webm}
    \cite{galaxy-rotation}
  \end{center}
\end{frame}

\note{
  \begin{itemize}
  \item Let me illustrate this effect with a video.
  \item In the video you see a simulation of the rotation of a spiral galaxy.
  \item If there would be no dark matter, the galaxy rotation would
    look like in the right panel: The outer stars are relatively
    slowly moving. This is not what one usually finds in rotating
    galaxies!
  \item Usually one finds that the outer stars move much faster, see
    left panel. This can be explained with the existance of Dark
    Matter (red) which fills the galaxy and changes the mass
    distribution.
  \end{itemize}
}

%%%%%%%%%%%%%%%%%%%%%%%%%%%%%%%%%%%%%%%%%%%%%%%%%%

\section{Magnetism}

\begin{frame}{\insertsection}
  \begin{center}
    \includegraphics[width=0.9\textwidth]{img/bar_magnet_foto}
    \footnotesize\cite{bar-magnet}
  \end{center}
\end{frame}

\note{
  \begin{itemize}
  \item To shed some light on Dark Matter, we'll start our journey
    with a conventional solid magnet (paramagnetism or
    ferromagnetism).
  \item We'll now look at a very interesting pheomenon
    regarding magnetism, which is called the ``anomalous magnetic
    moment''.
  \end{itemize}
}

%%%%%%%%%%%%%%%%%%%%%%%%%%%%%%%%%%%%%%%%%%%%%%%%%%

\begin{frame}{\insertsection: magnetic moment $\vec{m}$}
  \begin{tikzpicture}
    \only<1>{\magnet{1.8}{0.5}{3}}%
    \only<2>{\magnet{1.8}{0.5}{4}}%
    \only<3>{\magnet{1.8}{0.5}{5}}%
    \only<4>{\magnet{1.8}{0.5}{6}}%
    \only<5>{\magnet{1.8}{0.5}{7}}%
    \only<6>{\magnet{1.8}{0.5}{8}}%
    \only<7>{\magnet{1.8}{0.5}{9}}%
    \only<8>{\magnet{1.8}{0.5}{10}}%
    \only<9>{\magnet{1.8}{0.5}{11}}%
  \end{tikzpicture}  
\end{frame}

\note{
  \begin{itemize}
  \item This solid magnet has a magnetic field, which can be
    visualized with magnetic field lines (gray).
  \item If one would bring another magnet close to this magnet, the
    magnet would start experience a torque (Drehmoment).
  \item To describe the torque that the magnet experiences, one
    assigns a vector to the magnet, called the ``magnetic dipole
    moment'' $\vec{m}$ (blue).
  \item The magnetic dipole moment $\vec{m}$ of this shown magnet
    points from its south pole to its north pole.
  \item The magnitude of the magnetic dipole moment $\vec{m}$ (i.e.\
    the length of the vector) is a measure for the ``strength'' of the
    magnetic field.
  \end{itemize}
}

%%%%%%%%%%%%%%%%%%%%%%%%%%%%%%%%%%%%%%%%%%%%%%%%%%

% \begin{frame}{\insertsection}
%   Magnetisches Moment $\vec{m}$:
%   \begin{itemize}
%   \item Beschreibt WW eines magnetischen Dipols mit einem Magnetfeld
%   \item Betrag: "`Stärke der WW"'
%   \item Richtung: Angestrebte Ausrichtung zum Magnetfeld
%   \end{itemize}
%   Ursache des Magnetismus bei einem Permanentmagneten:
%   \begin{itemize}
%   \item Bahndrehimpuls
%   \item Spin
%   \end{itemize}
%   der Atome
% \end{frame}

%%%%%%%%%%%%%%%%%%%%%%%%%%%%%%%%%%%%%%%%%%%%%%%%%%

\begin{frame}{\insertsection: origin}
  \begin{center}
    \begin{tikzpicture}
      \def\lmag{1.8}  % length of magnet
      \def\wmag{0.5}  % thickness of magnet
      \coordinate (A) at (-\lmag/2,\wmag/2);
      \coordinate (B) at (\lmag/2,-\wmag/2);
      \draw[fill, color=green](A) rectangle ++(\lmag/2,-\wmag) node[black,midway]{S};
      \draw[fill, color=red](0,-\wmag/2) rectangle ++(\lmag/2,\wmag) node[black,midway]{N};
    \end{tikzpicture}
  \end{center}
\end{frame}

\note{
  \begin{itemize}
  \item In order to understand the origin of the magnetic moment, we
    must look closer into the magnet.
  \end{itemize}
}

%%%%%%%%%%%%%%%%%%%%%%%%%%%%%%%%%%%%%%%%%%%%%%%%%%

\begin{frame}{\insertsection: origin}
  \begin{center}
    \begin{tikzpicture}
      \def\lmag{4}  % length of magnet
      \def\wmag{1}  % thickness of magnet
      \coordinate (A) at (-\lmag/2,\wmag/2);
      \coordinate (B) at (\lmag/2,-\wmag/2);
      \draw[fill, color=green](A) rectangle ++(\lmag/2,-\wmag) node[black,midway]{\large S};
      \draw[fill, color=red](0,-\wmag/2) rectangle ++(\lmag/2,\wmag) node[black,midway]{\large N};
      %
      \foreach \X in {-1.9,-1.7,...,1.9} {%
        \foreach \Y in {-0.4,-0.2,...,0.4} {%
          \draw[fill,black] (\X,\Y) circle (0.01);
        }
      }
    \end{tikzpicture}
  \end{center}
\end{frame}

%%%%%%%%%%%%%%%%%%%%%%%%%%%%%%%%%%%%%%%%%%%%%%%%%%

\begin{frame}{\insertsection: origin}
  \begin{center}
    \begin{tikzpicture}
      \def\lmag{8}  % length of magnet
      \def\wmag{2}  % thickness of magnet
      \coordinate (A) at (-\lmag/2,\wmag/2);
      \coordinate (B) at (\lmag/2,-\wmag/2);
      \draw[fill, color=green](A) rectangle ++(\lmag/2,-\wmag) node[black,midway]{\Huge S};
      \draw[fill, color=red](0,-\wmag/2) rectangle ++(\lmag/2,\wmag) node[black,midway]{\Huge N};
      %
      \foreach \X in {-3.8,-3.4,...,3.9} {%
        \foreach \Y in {-0.8,-0.4,...,0.9} {%
          \draw[fill,black] (\X,\Y) circle (0.02);
        }
      }
    \end{tikzpicture}
  \end{center}
\end{frame}

\note{
  \begin{itemize}
  \item We see that the magnet consists of atoms.
  \end{itemize}
}

%%%%%%%%%%%%%%%%%%%%%%%%%%%%%%%%%%%%%%%%%%%%%%%%%%

\begin{frame}{\insertsection: origin}
  \begin{center}
    \begin{tikzpicture}
      \def\lmag{8}  % length of magnet
      \def\wmag{2}  % thickness of magnet
      \coordinate (A) at (-\lmag/2,\wmag/2);
      \coordinate (B) at (\lmag/2,-\wmag/2);
      \draw[fill, color=green](A) rectangle ++(\lmag/2,-\wmag) node[black,midway]{\Huge S};
      \draw[fill, color=red](0,-\wmag/2) rectangle ++(\lmag/2,\wmag) node[black,midway]{\Huge N};
      %
      \foreach \X in {-3.8,-3.4,...,3.9} {%
        \foreach \Y in {-0.8,-0.4,...,0.9} {%
          % \draw[fill,black] (\X,\Y) circle (0.02);
          \draw[black,->] (\X-0.1,\Y) -- ++(0.2,0);
        }
      }
    \end{tikzpicture}
  \end{center}
\end{frame}

\note{
  \begin{itemize}
  \item If we look at the atoms further, we find that each atom has a
    magnetic moment.
  \item The magnetic moments of all atoms together make up the total
    magnetic moment of the magnet.
  \end{itemize}
}

%%%%%%%%%%%%%%%%%%%%%%%%%%%%%%%%%%%%%%%%%%%%%%%%%%

\begin{frame}{\insertsection: origin}
  \begin{center}
    \begin{tikzpicture}
      \fill[even odd rule,inner color=blue,outer color=solarizedRebase03!100] (0,0) circle (3);
      \coordinate (n) at (0,0);
      \coordinate (e) at (2,0.5);
      \draw[fill,red] (n) circle (0.1);
      \draw[fill,green] (e) circle (0.05);
      \only<1>{%
        \node[below,black] at (n) {nucleus};
        \node[below,green] at (e) {electron};
      }%
      \only<2>{%
        \draw[very thick,red,->] ($(n)+(0,-1)$) -- ++(0,2);
        \draw[very thick,green,->] ($(e)+(0,-0.5)$) -- ++(0,1);
      }%
      \only<1-2>{%
        \draw[ultra thick,blue,->] (5,-1.5) -- node[right] {$\vec{m}$} ++(0,3);
      }%
    \end{tikzpicture}
  \end{center}
\end{frame}

\note{
  \begin{itemize}
  \item The next question is: Why does the atom have a magnetic
    moment?
  \item Atoms consist of a nucleous (made of protons and neutrons) and
    electrons, which are bound with the nucleous.
  \item The magnetic moment of an atom has three contributions:
    \begin{itemize}
    \item the magnetic moment of the electron(s)
    \item the magnetic moment of the nucleous
    \item the ``motion'' of the electron(s) around the nucleous
    \end{itemize}
  \item What we will look at in the following is the magnetic moment
    of the electron.
  \end{itemize}
}

%%%%%%%%%%%%%%%%%%%%%%%%%%%%%%%%%%%%%%%%%%%%%%%%%%

\begin{frame}{Electron $g$-factor}
  Magnetic moment of the electron:
  \begin{align*}
    \vec{m}_e &= g_e \frac{e}{2m_e}\vec{S}
  \end{align*}
  Measurement vs.\ prediction from classical Quantum
  Mechanics:
  \begin{align*}
    g_e^\Exp &= \num{2.00231930436146}(58) \\
    \pause
    g_e^{\text{QM}} &= 2
  \end{align*}
  Gigantic disagreement!
\end{frame}

% \begin{frame}{Electron $g$-factor}
%   Anomalous magnetic moment (relative deviation to $2$):
%   \begin{align*}
%     a_e = \frac{g_e-2}{2}
%   \end{align*}
%   \begin{align*}
%     a_e^\Exp &= \num{0.00115965218073}(28) \\
%     a_e^\SM  &= \num{0.00115965218164}(77)
%   \end{align*}
% \end{frame}

\note{
  \begin{itemize}
  \item The magnetic moment of an electron originates from its spin $\vec{S}$.
  \item The spin is a quantum property of the electron (besides its
    charges), which is known.
  \item The value of the magnetic moment of the electron is given by
    the elementary charge $e$, its mass $m_e$ and the so-called
    $g$-factor.
  \item The value of $g_e$ for the electron has been measured with
    extreme precision with a relative uncertainty of
    $\num{3e-13}$.
  \item The classical quantum mechanics predicts for the electron
    $g_e=2$, which is a gigantic disagreement!
  \item To understand how the sea of virtual particles changes the
    effective value of $g_e$ we have to look at the mechanism that
    describes the interaction between the electron's magnetic moment
    and an external magnet.
  \end{itemize}
}

\begin{frame}{Electron $g$-factor: Quantum Corrections}
  \begin{center}
    \begin{tikzpicture}
      \draw[fill,green] (0,0) coordinate (e) circle (0.1) node[left] {$e$};
      \draw[very thick,green,->] ($(e)+(0,-0.5)$) -- ++(0,1);
      \coordinate (M) at (4,2);
      \draw[yellow, photon] (M) -- node[below right] {$\gamma$} (e);
      \begin{scope}[shift={(M)},rotate={atan(2/4)}]
        \def\lmag{1.8}  % length of magnet
        \def\wmag{0.5}  % thickness of magnet
        \draw[fill, color=green] (0,\wmag/2) rectangle ++(-\lmag/2,-\wmag) node[black,midway,transform shape]{S};
        \draw[fill, color=red]   (0,\wmag/2) rectangle ++(+\lmag/2,-\wmag) node[black,midway,transform shape]{N};
      \end{scope}
      \only<2->{
      \QC{(1,1)}{rand*180};
      \QC{(2,1)}{rand*180};
      \QC{(1,2)}{rand*180};
      \QC{(1,-1)}{rand*180};
      \QC{(3,-1)}{rand*180};
      \QC{(1,-2)}{rand*180};
      \QC{(-1.5,2)}{rand*180};
      \QC{(-1.5,1)}{rand*180};
      \QC{(-2.5,2)}{rand*180};
      \QC{(-1,-1)}{rand*180};
      \QC{(-3,-1)}{rand*180};
      \QC{(-1,-2)}{rand*180};
      \QC{(0,-3)}{rand*180};
      \QC{(3,0)}{rand*180};
      \QC{(-3,0)}{rand*180};
      \QC{(4,-3)}{rand*180};
      \QC{(-4,-3)}{rand*180};
      % \foreach \N in {1,...,20} {%
      %   \QC{(rand*4,rand*3)}{rand*180};
      % }
      }
    \end{tikzpicture}
  \end{center}
\end{frame}

\note{
  \begin{itemize}
  \item The currently best theory that describes the interaction of an
    electron with an external magnet is the Standard Model of Particle
    Physics (SM).
  \item In the SM this interaction is described by the exchange of
    photons $\gamma$.
  \item However, one (i.e.\ a magnet) does never directly interact
    with the electron, because all the time virtual particles appear
    and dissappear.
  \item These virtual particles in particular surround the electron
    all the time.
  \item So, one (a magnet) interacts with an electron only through a
    ``sea'' of virtual particles.
  \item The existence of this ``sea'' of virtual particles changes the
    effective value of $g_e$ from the value $2$.
  \item So, to correctly predict $g_e$ one has to take into account
    the interaction of the photon with the ``sea'' of virtual
    particles, so-called quantum corrections.
  \end{itemize}
}

%%%%%%%%%%%%%%%%%%%%%%%%%%%%%%%%%%%%%%%%%%%%%%%%%%

\section{The Standard Model of Particle Physics}

\begin{frame}{\insertsection}
  \begin{center}
    \SMtable
  \end{center}
\end{frame}

\note{
  \begin{itemize}
  \item To calculate the quantum corrections we have to take into
    account all known particles.
  \item All known particles and their interactions are described by
    the SM.
  \item So, to predict $g_e$ one must take into account all known
    particles from the SM which make up the ``sea'' of virtual
    particles.
  \item The known particles described by the SM are: Electron, Muon,
    Tau, Quarks, Gauge bosons and Higgs boson.
  \end{itemize}
}

%%%%%%%%%%%%%%%%%%%%%%%%%%%%%%%%%%%%%%%%%%%%%%%%%%

\begin{frame}{Electron $g$-factor: Quantum Corrections}
  Direct interaction of an electron with a magnetic field (mediated by
  a photon):
  \begin{align*}
    \begin{tikzpicture}[thick,baseline={(0,0)}]
      \coordinate (a) at (0,0);
      \draw[photon] (0,1) node[above] {$\gamma$} -- (a);
      \draw[->-] (-1,-1) node[below left] {$e$} -- (a);
      \draw[->-] (a) -- (1,-1) node[below right] {$e$};
      \draw[fill] (a) circle (0.05); 
    \end{tikzpicture}
    && &\Rightarrow & g_e &= 2
  \end{align*}
\end{frame}

\note{
  \begin{itemize}
  \item The interaction of an electron with a photon can be described
    with so-called Feynman diagrams.
  \item The simplest diagram looks like this.
  \item It describes the ``direct'' interaction of a magnetic field
    with an electron (mediated by a photon)
  \item From this diagram one yields the classical quantum mechanics
    prediction $g_e=2$
  \end{itemize}
}

%%%%%%%%%%%%%%%%%%%%%%%%%%%%%%%%%%%%%%%%%%%%%%%%%%

\begin{frame}{Electron $g$-factor: Quantum Corrections}
  Next order (1-loop) quantum correction:
  \begin{align*}
    \begin{tikzpicture}[thick,baseline={(0,0)}]
      \coordinate (a) at (0,0);
      \coordinate (i1) at (-1,0);
      \coordinate (i2) at (1,0);
      \draw[photon] (0,1) -- (a);
      \draw[->-] (-2,0) -- (i1);
      \draw[] (i1) -- (a);
      \draw[] (a) -- (i2);
      \draw[->-] (i2) -- (2,0);
      \draw[photon] (i1) to[bend right=60] (i2);
      \draw[fill] (a) circle (0.05); 
      \draw[fill] (i1) circle (0.05); 
      \draw[fill] (i2) circle (0.05); 
    \end{tikzpicture}
    && g_e^{\text{1-loop}} &\approx \num{2.00232282} \\
    % && a_e^{\text{1-loop}} &\approx \num{0.00116141} \\
    &&&~\text{[Schwinger 1948]}
    % alpha = 1/137.035999206
    % g[a_] := 2 a + 2
    % g[0 + h alpha/(2 Pi) + h^2 (alpha/Pi)^2 (197/144 + Pi^2/12 + 3/4 Zeta[3] - Pi^2/2 Log[2])] // N[#,17]& // Expand
  \end{align*}
  Relative deviation:
  \begin{align*}
    a_e^{\text{1-loop}} := \frac{g_e^{\text{1-loop}}-2}{2} &\approx \num{0.00116141}
  \end{align*}
\end{frame}

\note{
  \begin{itemize}
  \item The interaction of the photon with the ``sea'' of virtual
    particles leads to so-called quantum corrections.
  \item The simplest diagram which depicts such a quantum correction
    contains one ``loop'' and has been calculated by Schwinger in
    1948.
  \item This quantum correction yields to a deviation from the value
    $g_e=2$.
  \item This relative deviation is called the ``anomalous magnetic
    moment'' and denoted as $a_e$.
  \end{itemize}
}

%%%%%%%%%%%%%%%%%%%%%%%%%%%%%%%%%%%%%%%%%%%%%%%%%%

\begin{frame}{Electron $g$-factor: Quantum Corrections}
  \begin{center}
    \begin{tikzpicture}
      \node[anchor=south west,inner sep=0] (image) at (0,0) {\includegraphics[width=0.8\textwidth]{img/Julian_Schwinger_headstone}};
      \begin{scope}[x={(image.south east)},y={(image.north west)}]
        \node[black] at (0.35,0.74) {\Large $a_e~=$};
      \end{scope}
    \end{tikzpicture}
    \cite{schwinger-headstone}
  \end{center}
\end{frame}

%%%%%%%%%%%%%%%%%%%%%%%%%%%%%%%%%%%%%%%%%%%%%%%%%%

\begin{frame}{Electron $g$-factor: Quantum Corrections}
  Quantum corrections with 2 loops:

  \bigskip
  \begin{tabular}{ccc}
    \begin{tikzpicture}[thick,baseline={(0,0)},scale=0.5]
      \coordinate (a) at (0,0);
      \coordinate (i1) at (-2,0);
      \coordinate (i2) at (2,0);
      \coordinate (i3) at (-1,0);
      \coordinate (i4) at (1,0);
      \draw[photon] (0,2) -- (a);
      \draw[] (-3,0) -- (3,0);
      \draw[photon] (i1) to[bend right=60] (i2);
      \draw[photon] (i3) to[bend right=60] (i4);
      \draw[fill] (a) circle (0.05);
      \draw[fill] (i1) circle (0.05);
      \draw[fill] (i2) circle (0.05);
      \draw[fill] (i3) circle (0.05);
      \draw[fill] (i4) circle (0.05);
    \end{tikzpicture}
    &
    \begin{tikzpicture}[thick,baseline={(0,0)},scale=0.5]
      \coordinate (a) at (0,0);
      \coordinate (i1) at (-2,0);
      \coordinate (i2) at (1,0);
      \coordinate (i3) at (-3,0);
      \coordinate (i4) at (-1,0);
      \draw[photon] (0,2) -- (a);
      \draw[] (-4,0) -- (2,0);
      \draw[photon] (i1) to[bend right=60] (i2);
      \draw[photon] (i3) to[bend left=60] (i4);
      \draw[fill] (a) circle (0.05);
      \draw[fill] (i1) circle (0.05);
      \draw[fill] (i2) circle (0.05);
      \draw[fill] (i3) circle (0.05);
      \draw[fill] (i4) circle (0.05);
    \end{tikzpicture}
    &
    \begin{tikzpicture}[thick,baseline={(0,0)},scale=0.5]
      \coordinate (a) at (0,0);
      \coordinate (i1) at (-3,0);
      \coordinate (i2) at (1,0);
      \coordinate (i3) at (-2,0);
      \coordinate (i4) at (-1,0);
      \draw[photon] (0,2) -- (a);
      \draw[] (-4,0) -- (2,0);
      \draw[photon] (i1) to[bend right=60] (i2);
      \draw[photon] (i3) to[bend left=60] (i4);
      \draw[fill] (a) circle (0.05);
      \draw[fill] (i1) circle (0.05);
      \draw[fill] (i2) circle (0.05);
      \draw[fill] (i3) circle (0.05);
      \draw[fill] (i4) circle (0.05);
    \end{tikzpicture}
    \\[1cm]
    \begin{tikzpicture}[thick,baseline={(0,0)},scale=0.5]
      \coordinate (a) at (0,0);
      \coordinate (i1) at (-1,0);
      \coordinate (i2) at (1,0);
      \coordinate (i3) at (-3,0);
      \coordinate (i4) at (-2,0);
      \draw[photon] (0,2) -- (a);
      \draw[] (-4,0) -- (2,0);
      \draw[photon] (i1) to[bend right=60] (i2);
      \draw[photon] (i3) to[bend left=60] (i4);
      \draw[fill] (a) circle (0.05);
      \draw[fill] (i1) circle (0.05);
      \draw[fill] (i2) circle (0.05);
      \draw[fill] (i3) circle (0.05);
      \draw[fill] (i4) circle (0.05);
    \end{tikzpicture}
    &
    \begin{tikzpicture}[thick,baseline={(0,0)},scale=0.5]
      \coordinate (a) at (0,0);
      \coordinate (i1) at (-2,0);
      \coordinate (i2) at (1,0);
      \coordinate (i3) at (-1,0);
      \coordinate (i4) at (2,0);
      \draw[photon] (0,2) -- (a);
      \draw[] (-3,0) -- (3,0);
      \draw[photon] (i1) to[bend right=60] (i2);
      \draw[photon] (i3) to[bend right=60] (i4);
      \draw[fill] (a) circle (0.05);
      \draw[fill] (i1) circle (0.05);
      \draw[fill] (i2) circle (0.05);
      \draw[fill] (i3) circle (0.05);
      \draw[fill] (i4) circle (0.05);
    \end{tikzpicture}
    &
    \begin{tikzpicture}[thick,baseline={(0,0)},scale=0.5]
      \coordinate (a) at (0,0);
      \coordinate (i1) at (-2,0);
      \coordinate (i2) at (2,0);
      \coordinate (i3) at (0,1);
      \coordinate (i4) at (0,2);
      \draw[photon] (0,3) -- (i4) (i3) -- (a);
      \draw[] (-3,0) -- (3,0);
      \draw[] (i3) to[bend right=90] (i4)
              (i4) to[bend right=90] (i3);
      \draw[photon] (i1) to[bend right=60] (i2);
      \draw[fill] (a) circle (0.05);
      \draw[fill] (i1) circle (0.05);
      \draw[fill] (i2) circle (0.05);
      \draw[fill] (i3) circle (0.05);
      \draw[fill] (i4) circle (0.05);
    \end{tikzpicture}
  \end{tabular}
  \begin{align*}
    a_e^{\text{2-loop}} &\approx \num{-0.00000177231}
  \end{align*}
\end{frame}

\note{
  \begin{itemize}
  \item The next higher order quantum corrections look like this.
  \item Eventually theorists must calculate millions of these
    so-called Feyman diagrams.
  \item They represent extremely complicated integrals to be solved.
  \end{itemize}
}

%%%%%%%%%%%%%%%%%%%%%%%%%%%%%%%%%%%%%%%%%%%%%%%%%%

\section{Electron $g$-factor}

\begin{frame}{\insertsection}
  Comparison measurement vs.\ multi-loop prediction for $a_e$:
  \begin{align*}
    a_e^\Exp &= (\numaeExp\pm \num{\DaeExp})\times \num{e-13} \\
    a_e^\SM  &= (\numaeSM \pm \num{\DaeSM})\times \num{e-13}
  \end{align*}
  \pause
  Agreement within a relative uncertainty of \textcolor{red}{$\approx\num{e-10}$}
\end{frame}

\note{
  \begin{itemize}
  \item The agreement between the experimental result and the
    prediction is excellent and within a relative uncertainty of
    \textcolor{red}{$\approx\num{e-10}$}.
  \item $a_e$ is one of the most precisely measured quantities in the
    world.
  \end{itemize}
}

%%%%%%%%%%%%%%%%%%%%%%%%%%%%%%%%%%%%%%%%%%%%%%%%%%

\begin{frame}{\insertsection}
  \begin{columns}
    \column{0.7\textwidth}
    \begin{center}
      \begin{tikzpicture}
        \begin{axis}[
          width=\textwidth,
          height=0.8\textheight,
          xlabel = {$(a_e - a_e^\SM)\times 10^{13}$},
          ymajorticks = false,
          xmin = -20, xmax = 20,
          ymin = 0, ymax = 4,
          ]
          \draw[fill,green] (-\DaeSM, 0) rectangle (\DaeSM, 4);
          \addplot[blue,only marks,mark=*,error bars/.cd,x dir=both, x explicit] coordinates {
            % (\aeExp  - \aeSM, 0.5) +- (\DaeExp , 0)
            (-9.1, 0.5) +- (\DaeExp , 0)
          };
          \node[above,blue] at (-9.1, 0.5) {Experiment};
          \node[black,rotate=90] at (0, 2) {SM prediction};
        \end{axis}
      \end{tikzpicture}
    \end{center}

    \column{0.29\textwidth}
    $a_e^{\Exp} - a_e^\SM = (-9.1 \pm 8.2)\times 10^{-13}$
    % $a_e^{\Exp} - a_e^\SM \approx (-9 \pm 9)\times 10^{-13}$

  \end{columns}
\end{frame}

\note{
  \begin{itemize}
  \item Visually, the deviation is compatible with ``no deviation''.
  \end{itemize}
}

%%%%%%%%%%%%%%%%%%%%%%%%%%%%%%%%%%%%%%%%%%%%%%%%%%

\section{Muon $g$-factor}

\begin{frame}{\insertsection}
  \begin{center}
    \SMtable
  \end{center}
\end{frame}

\note{
  \begin{itemize}
  \item So far we have looked at the magnetic moment of the electron.
  \item Let's look at the magnetic moment of its heavier cousin, the
    muon.
  \item The muon has the same properties as the electron, except for
    its mass, which is approximately $200$ times heavier.
  \end{itemize}
}

%%%%%%%%%%%%%%%%%%%%%%%%%%%%%%%%%%%%%%%%%%%%%%%%%%

\begin{frame}{\insertsection}
  Standard Model multi-loop prediction:
  \begin{align*}
    a_\mu^\SM &= (\numamuSM \pm \DamuSM)\times 10^{-10} \\
    a_\mu^\Exp &= ~ ?
  \end{align*}
\end{frame}

\note{
  \begin{itemize}
  \item If one carries out the same calculation for the muon as for
    the electron, one arrives at the following prediction for the
    relative deviation of the magnetic moment of the muon from the
    value 2.
  \item The next question is: What is the actual value? Let's measure!
  \item There are two experiments, which have performed measurements
    of $a_\mu$: BNL (Brookhaven National Laboratory, USA) and FNAL
    (Fermi National Accelerator Laboratory, USA). The experiment at
    FNAL is currently running and not finished.
  \end{itemize}
}

%%%%%%%%%%%%%%%%%%%%%%%%%%%%%%%%%%%%%%%%%%%%%%%%%%

\section{Measurement}

%%%%%%%%%%%%%%%%%%%%%%%%%%%%%%%%%%%%%%%%%%%%%%%%%%

\begin{frame}{\insertsection: Fermilab (FNAL)}
  \begin{center}
    \includegraphics[width=0.6\textwidth]{img/FNAL-14-0280-01D}
    \footnotesize [FNAL]
  \end{center}
\end{frame}

\note{
  \begin{itemize}
  \item To measure $a_\mu$ we need to study muons in great detail
  \item However, muons are difficult to produce and tend to decay very
    quickly (mean life time: $\approx\SI{2}{\us}$)
  \item To create and study them we need a particle accelerator
    (FNAL)!
    \begin{itemize}
    \item Take a beam of protons $p^+$
    \item Let them hit a target, where they produce pions $\pi^+$
    \item The pions decay into muons $\mu^+$.
    \end{itemize}
  \item To study the muons, we must bring them to a high speed (close
    to the speed of light), so that we have more time to study them
    (time dillatation)
  \item We need a ring!
  \end{itemize}
}

%%%%%%%%%%%%%%%%%%%%%%%%%%%%%%%%%%%%%%%%%%%%%%%%%%

\begin{frame}{\insertsection: BNL}
  \begin{center}
    \includegraphics[width=0.8\textwidth]{img/BNL}
    \footnotesize \cite{BNL-ring}
  \end{center}
\end{frame}

\note{
  \begin{itemize}
  \item There has been a predecessor experiment at the Brookhaven
    National Laboratory (BNL).
  \item Idea: Reuse their ring!
  \item Problem: Transportation of the ring so that it does not get
    distrorted (less than \SI{3}{\mm} tolerance)
  \end{itemize}
}

%%%%%%%%%%%%%%%%%%%%%%%%%%%%%%%%%%%%%%%%%%%%%%%%%%

\begin{frame}{\insertsection: at Fermilab (FNAL)}
  \begin{center}
    \includegraphics[width=0.8\textwidth]{img/ring-transport}
    \footnotesize \cite{Transport2}
  \end{center}
\end{frame}

\note{
  \begin{itemize}
  \item ``Solution'': Ship to a large part over water to avoid bridges
    and roads
  \item The transportation was extremely challenging and took over a
    month.
  \end{itemize}
}

%%%%%%%%%%%%%%%%%%%%%%%%%%%%%%%%%%%%%%%%%%%%%%%%%%

\begin{frame}{\insertsection: at Fermilab (FNAL)}
  \begin{center}
    \includegraphics[width=0.8\textwidth]{img/FNAL-20130719-St-Louis-Arch}
    \footnotesize \cite{Transport3}
  \end{center}
\end{frame}

%%%%%%%%%%%%%%%%%%%%%%%%%%%%%%%%%%%%%%%%%%%%%%%%%%

\begin{frame}{\insertsection: at Fermilab (FNAL)}
  \begin{center}
    \includegraphics[width=0.8\textwidth]{img/FNAL-13-0221-08D}
    \footnotesize \cite{Transport1}
  \end{center}
\end{frame}

\note{
  \begin{itemize}
  \item Let me show you a video.
  \end{itemize}
}

%%%%%%%%%%%%%%%%%%%%%%%%%%%%%%%%%%%%%%%%%%%%%%%%%%

\begin{frame}{\insertsection: Fermilab (FNAL)}
  \begin{center}
    \movie[width=0.9\textwidth,autostart,loop,showcontrols]{\includegraphics[width=0.9\textwidth]{videos/Muon-g-2-moving.png}}{videos/Muon-g-2-moving.mp4}
    \cite{muon-g-2-moving}
  \end{center}
\end{frame}

%%%%%%%%%%%%%%%%%%%%%%%%%%%%%%%%%%%%%%%%%%%%%%%%%%

\begin{frame}{\insertsection: Fermilab (FNAL)}
  \begin{center}
    \includegraphics[width=0.9\textwidth]{img/FNAL-17-0188-17}
    \footnotesize \cite{FNAL-ring}
  \end{center}
\end{frame}

\note{
  \begin{itemize}
  \item Idea: The polarized muons $\mu^+$ circle around the muon
    storage ring
  \end{itemize}
}

%%%%%%%%%%%%%%%%%%%%%%%%%%%%%%%%%%%%%%%%%%%%%%%%%%

\begin{frame}{\insertsection: Fermilab (FNAL)}
  \begin{center}
    \colorbox{white}{\includegraphics[width=0.6\textwidth]{img/FNAL-magnet}}
    \footnotesize \cite{Miller:2012opa}
    \hfill
    \begin{tikzpicture}[thick]
      \pgfmathsetmacro{\leng}{1.5}
      \draw[->,green] (-45:\leng) -- ({90+45}:\leng) node[above] {$\vec{m}_\mu$};
      \draw[->,blue] (0,-\leng) -- (0,\leng) node[above] {$\vec{B}$};
      \draw[green,fill] (0,0) circle (0.5);
      \draw[->,dashed] (-45:\leng) arc [start angle=0, end angle=-260, x radius=1cm, y radius=0.2cm];
    \end{tikzpicture}
  \end{center}
\end{frame}

\note{
  \begin{itemize}
  \item Within the ring there is a magnetic field, which lets the
    muon's spins rotate (Lamor precession).
  \end{itemize}
}

%%%%%%%%%%%%%%%%%%%%%%%%%%%%%%%%%%%%%%%%%%%%%%%%%%

\begin{frame}{\insertsection}
  Measure the deviation of the muon's \textcolor{red}{spin precession
    frequency} from the \textcolor{green}{cyclotron frequency}:
  \begin{center}
    \begin{tikzpicture}[very thick]
      \pgfmathsetmacro{\radi}{2};
      \pgfmathsetmacro{\leng}{0.8};
      \draw (0,0) circle (2);
      \foreach \N in {0,1,2,3} {%
        \draw[fill]     ({-\N*90}:\radi) circle (0.1);
        \draw[->,green] ({-\N*90}:\radi) -- ++({-(\N+1)*90}:\leng);
        \draw[->,red]   ({-\N*90}:\radi) -- ++({-(\N+1-0.2*\N)*90}:\leng);
      }
    \end{tikzpicture}
  \end{center}
\end{frame}

\note{
  \begin{itemize}
  \item Idea: align the magnetic field so that if $g_\mu=2$
    ($a_\mu=0$), the muon's spin would point exactly into the muon's
    momentum direction ($\to$ rotation with cyclotron frequency
    $\omega_C$)
  \item Measure the deviation $\omega_a=\omega_S-\omega_C$ of the
    muon's spin precession frequency (Lamor frequency $\omega_S$) from
    the cyclotron frequency ($\omega_C$)
  \item $\omega_a$ depends only on the anomalous magnetic moment
    $a_\mu$
  \item $\omega_a$ depends linearly on the magnetic field $B$
  \item The magnetic field $B$ must be controlled/known to
    sub-part-per-million (ppm) precision
  \end{itemize}
}

%%%%%%%%%%%%%%%%%%%%%%%%%%%%%%%%%%%%%%%%%%%%%%%%%%

\begin{frame}{\insertsection: Fermilab (FNAL)}
  \begin{columns}
    \column{0.4\textwidth}
    Inside the ring the muons decay:
    \begin{align*}
      \mu^+ \to e^+ + \nu_e + \bar{\nu}_\mu
    \end{align*}
    Energy of $e^+$ depends on muon's spin direction
    % $\to$ infer deviation $\omega_a$
    $\to$ infer $a_\mu$

    \column{0.6\textwidth}
    \centering
    % left lower right upper
    \colorbox{white}{\includegraphics[trim={130 180 110 360},clip,width=\textwidth]{img/wiggly}}

    \cite{Muong-2:2015xgu}
  \end{columns}
\end{frame}

\note{
  \begin{itemize}
  \item At some point the muons begin to decay, mainly into positrons
    and neutrinos.
  \item Depending on the spin direction, their decay products have
    different energy.
  \item Measuring the energy of the decay products allows the
    experimentalists to derive the munon's spin precession frequency
    and thus the value of $a_\mu$.
  \end{itemize}
}

%%%%%%%%%%%%%%%%%%%%%%%%%%%%%%%%%%%%%%%%%%%%%%%%%%

\begin{frame}{\insertsection}
  Experimental measurements:
  \begin{align*}
    a_\mu^{\text{CERN}} &= (\numamuCERN \pm \DamuCERN)\times 10^{-10} && (1979)~\text{\cite{CERN-Mainz-Daresbury:1978ccd}}\\
    a_\mu^{\text{BNL}} &= (\numamuBNL \pm \DamuBNL)\times 10^{-10} && (2006)~\text{\cite{Muong-2:2006rrc}}\\
    a_\mu^{\text{FNAL}} &= (\numamuFNAL \pm \DamuFNAL)\times 10^{-10} && (2021)~\text{\cite{PhysRevLett.126.141801}}
    \intertext{Combined:}
    a_\mu^{\Exp} &= (\numamuExp \pm \DamuExp)\times 10^{-10} && (2021)~\text{\cite{PhysRevLett.126.141801}}
  \end{align*}
\end{frame}

\note{
  \begin{itemize}
  \item Here are the three measurements
  \item All measurements are compatible with each other
  \item Since these numbers are hard to read, let me visualize them.
  \end{itemize}
}

%%%%%%%%%%%%%%%%%%%%%%%%%%%%%%%%%%%%%%%%%%%%%%%%%%

\section{Comparison of measurement and prediction}

\begin{frame}{\insertsection}
  \begin{columns}
    \column{0.7\textwidth}
    \begin{center}
      \begin{tikzpicture}
        \begin{axis}[
          width=\textwidth,
          height=0.8\textheight,
          xlabel = {$(a_\mu - a_\mu^\SM)\times 10^{10}$},
          ymajorticks = false,
          xmin = -50, xmax = 150,
          ymin = 0, ymax = 4,
          ]
          \addplot[blue,dashed,only marks,mark=*,error bars/.cd,x dir=both, x explicit] coordinates {
            (\amuCERN - \amuSM, 3) +- (\DamuCERN, 0)
            (\amuBNL  - \amuSM, 2) +- (\DamuBNL , 0)
            (\amuFNAL - \amuSM, 1) +- (\DamuFNAL, 0)
          };
          \draw[fill,green] (-\DamuSM, 0) rectangle (\DamuSM, 5);
          \node[above] at (\amuCERN - \amuSM, 3) {CERN};
          \node[above] at (\amuBNL  - \amuSM, 2) {BNL};
          \node[above] at (\amuFNAL - \amuSM, 1) {FNAL};
          \node[above right,green,rotate=90] at (0, 0.5) {SM prediction};
        \end{axis}
      \end{tikzpicture}
    \end{center}

    \column{0.29\textwidth}
  \end{columns}
\end{frame}

\note{
  \begin{itemize}
  \item The first measurement from CERN is compatible with the SM
    prediction due to its large uncertainty.
  \item The refined measurements from BNL and FNAL have significantly
    smaller uncertainties.
  \item Question: How large is the difference between the combined
    measurement and the SM prediction?
  \end{itemize}
}

%%%%%%%%%%%%%%%%%%%%%%%%%%%%%%%%%%%%%%%%%%%%%%%%%%

\begin{frame}{\insertsection}
  \begin{columns}
    \column{0.7\textwidth}
    \begin{center}
      \begin{tikzpicture}
        \begin{axis}[
          width=\textwidth,
          height=0.8\textheight,
          xlabel = {$(a_\mu - a_\mu^\SM)\times 10^{10}$},
          ymajorticks = false,
          xmin = -10, xmax = 40,
          ymin = 0, ymax = 4,
          ]
          \addplot[blue,dashed,only marks,mark=*,error bars/.cd,x dir=both, x explicit] coordinates {
            (\amuBNL  - \amuSM, 3) +- (\DamuBNL , 0)
            (\amuFNAL - \amuSM, 2) +- (\DamuFNAL, 0)
          };
          \addplot[blue,only marks,mark=*,error bars/.cd,x dir=both, x explicit] coordinates {
            (\amuExp  - \amuSM, 1) +- (\DamuExp , 0)
          };
          \draw[fill,green] (-\DamuSM, 0) rectangle (\DamuSM, 4);
          \node[above] at (\amuBNL  - \amuSM, 3) {BNL};
          \node[above] at (\amuFNAL - \amuSM, 2) {FNAL};
          \node[above] at (\amuExp  - \amuSM, 1) {Experiment};
          \node[black,rotate=90] at (0, 2) {SM prediction};
        \end{axis}
      \end{tikzpicture}
    \end{center}

    \column{0.29\textwidth}
    % https://journals.aps.org/prl/abstract/10.1103/PhysRevLett.126.141801
    $a_\mu^{\Exp} - a_\mu^\SM = (25.1 \pm 5.9)\times 10^{-10}$

    \bigskip

    Deviation $\approx 4.2\sigma$

    \bigskip

    % using SpecialFunctions
    % DF(n) = erf(n/sqrt(2))
    % (1 - DF(4.2))*100

    $P(\text{data}|\SM)\approx \num{0.0027}\%$

  \end{columns}
\end{frame}

\note{
  \begin{itemize}
  \item The deviation between the SM prediction and the two
    experiments is $4.2\sigma$
  \item This can be interpreted as follows: Assume the SM is the
    correct model that describes the physics at that scale. The
    probability to observe this data is $\approx \num{0.0027}\%$.
  \item This is an extremely small probability, which suggests that
    the SM may not be the correct model to describe this physics.
  \item However, for physicists this is still not good enough. The
    conventin is: If the deviation is more than $5\sigma$
    ($0.00006\%$), then the underlying model shall be rejected.
  \item The current experiment at FNAL aims to be precise enough to
    reach the $5\sigma$ limit. More refined results are expected to
    be published within the next few years.
  \end{itemize}
}

%%%%%%%%%%%%%%%%%%%%%%%%%%%%%%%%%%%%%%%%%%%%%%%%%%

\section{Where does the deviation come from?}

\begin{frame}{\insertsection}
  \begin{center}
    \includegraphics[width=0.4\textwidth]{img/einstein-cut}
    \cite{einstein}
  \end{center}
  \pause
  Maybe there are more particles, which we have not observed yet?
\end{frame}

\note{
  \begin{itemize}
  \item Anyway, the deviation of $4.2\sigma$ is still very large!
  \item So, theorists are searching for explanations for this
    deviation.
  \item A possible explanation could be: Maybe there are more
    particles, which we have not observed yet, which contribute to the
    quantum corrections to $a_\mu$?
  \end{itemize}
}

%%%%%%%%%%%%%%%%%%%%%%%%%%%%%%%%%%%%%%%%%%%%%%%%%%

\begin{frame}{\insertsection}
  Maybe there are more Higgs bosons?
  \begin{center}
    \begin{tikzpicture}[node distance = 2.5em, auto]
      \node[quark] (u) {$u$};
      \node[quark, below of=u] (d) {$d$};
      \node[quark, right of=u] (c) {$c$};
      \node[quark, below of=c] (s) {$s$};
      \node[quark, right of=c] (t) {$t$};
      \node[quark, below of=t] (b) {$b$};
      \node[lepton, below of=d] (ne) {$\nu_e$};
      \node[lepton, below of=ne] (e) {$e$};
      \node[lepton, right of=ne] (nm) {$\nu_\mu$};
      \node[lepton, below of=nm] (m) {$\mu$};
      \node[lepton, right of=nm] (nt) {$\nu_\tau$};
      \node[lepton, below of=nt] (ta) {$\tau$};
      \node[gauge, right of=t] (gamma) {$\gamma$};
      \node[gauge, below of=gamma] (g) {$g$};
      \node[gauge, below of=g] (Z) {$Z$};
      \node[gauge, below of=Z] (W) {$W$};
      \node[scalar, right of=W] (h) {$h$};
      \node[scalar, above of=h] (H) {$H$};
      \node[scalar, above of=H] (A) {$A$};
      \node[scalar, above of=A] (Hp) {$H^\pm$};
      \node[rotate=90] (quarks)  at ($(u)!0.5!(d)+(-1,0)$)  {quarks};
      \node[rotate=90] (leptons) at ($(ne)!0.5!(e)+(-1,0)$) {leptons};
      \node[below of=h] (higgs) {Higgses};
      \node[above of=gamma, align=center] (gauge) {gauge\\[-0.5em] bosons};
    \end{tikzpicture}
  \end{center}
\end{frame}

\note{
  \begin{itemize}
  \item One well-motivated possibility for an extension of the SM with
    new particles is the so-called Two-Higgs-Doublet-Model (2HDM)
  \item It contains additional Higgs bosons
  \item Their contribution to $a_\mu$ can explain the deviation.
  \end{itemize}
}

%%%%%%%%%%%%%%%%%%%%%%%%%%%%%%%%%%%%%%%%%%%%%%%%%%

\subsection{Two-Higgs Doublet Model}

\begin{frame}{\insertsubsection}
  New quantum corrections in the 2HDM with 1 loop:
  \begin{align*}
    \begin{tikzpicture}[thick,baseline={(0,0)}]
      \pgfmathsetmacro{\angl}{60};
      \coordinate (a) at (0,0);
      \coordinate (i1) at ({180+\angl}:1);
      \coordinate (i2) at (-\angl:1);
      \draw[photon] (0,1) -- (a);
      \draw[->-] ({180+\angl}:2) node[below left] {$\mu$} -- (i1);
      \draw[dashed] (i1) -- node[above left] {$H^\pm$} (a);
      \draw[dashed] (a) -- (i2);
      \draw[->-] (i2) -- (-\angl:2) node[below right] {$\mu$};
      \draw[->-] (i1) -- (i2);
      \draw[fill] (a) circle (0.05); 
      \draw[fill] (i1) circle (0.05); 
      \draw[fill] (i2) circle (0.05); 
    \end{tikzpicture}
    +
    \begin{tikzpicture}[thick,baseline={(0,0)}]
      \pgfmathsetmacro{\angl}{60};
      \coordinate (a) at (0,0);
      \coordinate (i1) at ({180+\angl}:1);
      \coordinate (i2) at (-\angl:1);
      \draw[photon] (0,1) -- (a);
      \draw[->-] ({180+\angl}:2) node[below left] {$\mu$} -- (i1);
      \draw[->-] (i1) -- (a);
      \draw[->-] (a) -- (i2);
      \draw[->-] (i2) -- (-\angl:2) node[below right] {$\mu$};
      \draw[dashed] (i1) -- node[below] {$H$} (i2);
      \draw[fill] (a) circle (0.05); 
      \draw[fill] (i1) circle (0.05); 
      \draw[fill] (i2) circle (0.05); 
    \end{tikzpicture}
    +
    \begin{tikzpicture}[thick,baseline={(0,0)}]
      \pgfmathsetmacro{\angl}{60};
      \coordinate (a) at (0,0);
      \coordinate (i1) at ({180+\angl}:1);
      \coordinate (i2) at (-\angl:1);
      \draw[photon] (0,1) -- (a);
      \draw[->-] ({180+\angl}:2) node[below left] {$\mu$} -- (i1);
      \draw[->-] (i1) -- (a);
      \draw[->-] (a) -- (i2);
      \draw[->-] (i2) -- (-\angl:2) node[below right] {$\mu$};
      \draw[dashed] (i1) -- node[below] {$A$} (i2);
      \draw[fill] (a) circle (0.05); 
      \draw[fill] (i1) circle (0.05); 
      \draw[fill] (i2) circle (0.05); 
    \end{tikzpicture}
  \end{align*}
\end{frame}

\note{
  \begin{itemize}
  \item The additional Higgs bosons contribute to the quantum
    corrections to the magnetic moment.
  \item Their contribution (including 2-loop corrections) can explain
    the observed deviation of $a_\mu$.
  \end{itemize}
}

%%%%%%%%%%%%%%%%%%%%%%%%%%%%%%%%%%%%%%%%%%%%%%%%%%

\subsection{Minimal Supersymmetry}

\begin{frame}{\insertsubsection}
  Maybe there is a spin-partner for each particle?
  \begin{center}
    \begin{tikzpicture}[node distance = 2.5em, auto]
      \node[quark] (u) {$u$};
      \node[quark, below of=u] (d) {$d$};
      \node[quark, right of=u] (c) {$c$};
      \node[quark, below of=c] (s) {$s$};
      \node[quark, right of=c] (t) {$t$};
      \node[quark, below of=t] (b) {$b$};
      \node[lepton, below of=d] (ne) {$\nu_e$};
      \node[lepton, below of=ne] (e) {$e$};
      \node[lepton, right of=ne] (nm) {$\nu_\mu$};
      \node[lepton, below of=nm] (m) {$\mu$};
      \node[lepton, right of=nm] (nt) {$\nu_\tau$};
      \node[lepton, below of=nt] (ta) {$\tau$};
      \node[gauge, right of=t] (gamma) {$\gamma$};
      \node[gauge, below of=gamma] (g) {$g$};
      \node[gauge, below of=g] (Z) {$Z$};
      \node[gauge, below of=Z] (W) {$W$};
      \node[scalar, below of=e] (h) {$h$};
      \node[scalar, right of=h] (H) {$H$};
      \node[scalar, below of=h] (A) {$A$};
      \node[scalar, right of=A] (Hpm) {$H^\pm$};
      \node[rotate=90] (quarks)  at ($(u)!0.5!(d)+(-1,0)$)  {quarks};
      \node[rotate=90] (leptons) at ($(ne)!0.5!(e)+(-1,0)$) {leptons};
      \node[below=of A.west, anchor=west] (higgs) {Higgses};
      % \node[above of=gamma, align=center] (gauge) {Eich-\\ bosonen};
    \end{tikzpicture}
    \hfill
    \begin{tikzpicture}[node distance = 2.5em, auto]
      \node[squark] (u) {$\tilde u$};
      \node[squark, below of=u] (d) {$\tilde d$};
      \node[squark, right of=u] (c) {$\tilde c$};
      \node[squark, below of=c] (s) {$\tilde s$};
      \node[squark, right of=c] (t) {$\tilde t$};
      \node[squark, below of=t] (b) {$\tilde b$};
      \node[slepton, below of=d] (ne) {$\tilde \nu_e$};
      \node[slepton, below of=ne] (e) {$\tilde e$};
      \node[slepton, right of=ne] (nm) {$\tilde \nu_\mu$};
      \node[slepton, below of=nm] (m) {$\tilde \mu$};
      \node[slepton, right of=nm] (nt) {$\tilde \nu_\tau$};
      \node[slepton, below of=nt] (ta) {$\tilde \tau$};
      \node[gaugino, right of=t] (gamma) {$\tilde \gamma$};
      \node[gaugino, below of=gamma] (g) {$\tilde g$};
      \node[gaugino, below of=g] (Z) {$\tilde Z$};
      \node[gaugino, below of=Z] (W) {$\tilde W$};
      \node[higgsino, below of=e] (h) {$\tilde h_1^0$};
      \node[higgsino, right of=h] (H) {$\tilde h_2^0$};
      \node[higgsino, below of=h] (A) {$\tilde h^-_1$};
      \node[higgsino, below of=H] (Hpm) {$\tilde h^+_2$};
      \node[rotate=90] (quarks)  at ($(u)!0.5!(d)+(-1,0)$)  {squarks};
      \node[rotate=90] (leptons) at ($(ne)!0.5!(e)+(-1,0)$) {sleptons};
      \node[below=of A.west, anchor=west] (higgs) {higgsinos};
      \node[above of=gamma, align=center] (gauge) {gauginos};
    \end{tikzpicture}
  \end{center}
\end{frame}

\note{
  \begin{itemize}
  \item Another well-motivated possibility is Supersymmetry.
  \item In Supersymmetry for each particle there is a partner with a
    different spin.
  \item The contribution of the sleptons, gauginos and Higgsinos to
    $a_\mu$ can easily explain the deviation.
  \item Further advantages:
    \begin{itemize}
    \item The MSSM contains particles which are candidates for Dark
      Matter.
    \item The MSSM can (to some extent) explain the mass of the Higgs
      boson.
    \item The MSSM provides a way to unify the three forces to one
      universal force (aka Grand Unification).
    \item The MSSM can be connected to high-energy String Theory and
      theories that describe Quantum Gravity (aka Theory of
      Everything).
    \end{itemize}
  \item Problems: None of the new spin-partner particles have been
    observed yet. That is one reason the LHC was built. Searches for
    these particles are on-going.
  \end{itemize}
}

%%%%%%%%%%%%%%%%%%%%%%%%%%%%%%%%%%%%%%%%%%%%%%%%%%

\begin{frame}{\insertsubsection}
  New quantum corrections in the Minimal Supersymmetric Standard Model
  (MSSM) with 1 loop:
  \begin{align*}
    \begin{tikzpicture}[thick,baseline={(0,0)}]
      \pgfmathsetmacro{\angl}{60};
      \coordinate (a) at (0,0);
      \coordinate (i1) at ({180+\angl}:1);
      \coordinate (i2) at (-\angl:1);
      \draw[photon] (0,1) -- (a);
      \draw[->-] ({180+\angl}:2) node[below left] {$\mu$} -- (i1);
      \draw[->-] (i1) -- node[above left] {$\tilde{h}_{1,2}^\pm, \tilde{W}^\pm$} (a);
      \draw[->-] (a) -- (i2);
      \draw[->-] (i2) -- (-\angl:2) node[below right] {$\mu$};
      \draw[dashed] (i1) -- node[below] {$\tilde{\nu}$} (i2);
      \draw[fill] (a) circle (0.05); 
      \draw[fill] (i1) circle (0.05); 
      \draw[fill] (i2) circle (0.05); 
    \end{tikzpicture}
    +
    \begin{tikzpicture}[thick,baseline={(0,0)}]
      \pgfmathsetmacro{\angl}{60};
      \coordinate (a) at (0,0);
      \coordinate (i1) at ({180+\angl}:1);
      \coordinate (i2) at (-\angl:1);
      \draw[photon] (0,1) -- (a);
      \draw[->-] ({180+\angl}:2) node[below left] {$\mu$} -- (i1);
      \draw[->-,dashed] (i1) -- node[above left] {$\tilde{\mu}$} (a);
      \draw[->-,dashed] (a) -- (i2);
      \draw[->-] (i2) -- (-\angl:2) node[below right] {$\mu$};
      \draw[] (i1) -- node[below] {$\tilde{h}_{1,2}^0$} (i2);
      \draw[fill] (a) circle (0.05); 
      \draw[fill] (i1) circle (0.05); 
      \draw[fill] (i2) circle (0.05); 
    \end{tikzpicture}
    +
    \begin{tikzpicture}[thick,baseline={(0,0)}]
      \pgfmathsetmacro{\angl}{60};
      \coordinate (a) at (0,0);
      \coordinate (i1) at ({180+\angl}:1);
      \coordinate (i2) at (-\angl:1);
      \draw[photon] (0,1) -- (a);
      \draw[->-] ({180+\angl}:2) node[below left] {$\mu$} -- (i1);
      \draw[->-,dashed] (i1) -- node[above left] {$\tilde{\mu}$} (a);
      \draw[->-,dashed] (a) -- (i2);
      \draw[->-] (i2) -- (-\angl:2) node[below right] {$\mu$};
      \draw[] (i1) -- node[below] {$\tilde{\gamma}, \tilde{Z}$} (i2);
      \draw[fill] (a) circle (0.05); 
      \draw[fill] (i1) circle (0.05); 
      \draw[fill] (i2) circle (0.05); 
    \end{tikzpicture}
  \end{align*}
\end{frame}

\note{
  \begin{itemize}
  \item The additional spin-partners contribute to the quantum
    corrections to the magnetic moment.
  \item Their contribution can explain the observed deviation of
    $a_\mu$.
  \end{itemize}
}

%%%%%%%%%%%%%%%%%%%%%%%%%%%%%%%%%%%%%%%%%%%%%%%%%%

\section{Summary}

\begin{frame}{\insertsection}
  \begin{itemize}
  \item $a_\mu$ describes the interaction strength of the muon's spin
    with a magnetic field
  \item $a_\mu$ is governed by quantum corrections
  \item $4.2\sigma$ discrepancy between prediction and measurment
  \item[$\Rightarrow$] hints to new, unknown particles
  \end{itemize}
  \begin{center}
    Let's wait for more data!
  \end{center}
\end{frame}

%%%%%%%%%%%%%%%%%%%%%%%%%%%%%%%%%%%%%%%%%%%%%%%%%%

\section{Backup}

\begin{frame}{}
  \begin{center}
    \Large Backup
  \end{center}
\end{frame}

%%%%%%%%%%%%%%%%%%%%%%%%%%%%%%%%%%%%%%%%%%%%%%%%%%

\begin{frame}{Two-Higgs Doublet Model}
  \begin{center}
    \includegraphics[width=0.6\textwidth]{img/mA-tb-X}
    \cite{Athron:2021evk}
  \end{center}
\end{frame}

\note{
  \begin{itemize}
  \item Here is an example plot that shows the region (green) where
    the 2HDM can explain the observed value of $a_\mu$ as a function
    of the free parameters of the model.
  \end{itemize}
}

%%%%%%%%%%%%%%%%%%%%%%%%%%%%%%%%%%%%%%%%%%%%%%%%%%

\begin{frame}{Minimal Supersymmetry}
  \begin{center}
    \includegraphics[width=0.6\textwidth]{scans/MSSM/scan}
  \end{center}
\end{frame}

\note{
  \begin{itemize}
  \item Here is an example plot that shows the region (green) where
    the MSSM can explain the observed value of $a_\mu$ as a function
    of the two new particle masses of the model: the light
    spin-partner of the muon (the smuon) and the lightest neutralino
    (a Dark Matter candidate)
  \item From such studies one can derive relations (or even limits) on
    the spin-partner masses, given the measured value of $a_\mu$
  \end{itemize}
}

%%%%%%%%%%%%%%%%%%%%%%%%%%%%%%%%%%%%%%%%%%%%%%%%%%

\begin{frame}{Spin rotation frequencies}
  Lamor frequency:
  \begin{align*}
    \omega_S &= -g\frac{Qe}{2m}B - (1-\gamma)\frac{Qe}{\gamma m}B
  \end{align*}
  Cyclotron frequency:
  \begin{align*}
    \omega_C &= -\frac{Qe}{\gamma m}B
  \end{align*}
  Measure difference:
  \begin{align*}
    \omega_a &= \omega_S - \omega_C = -\frac{g-2}{2}\frac{Qe}{m}B = -a\frac{Qe}{m}B
  \end{align*}
\end{frame}

\note{
  \begin{itemize}
  \item Idea: measure the deviation $\omega_a$ of the muon's spin
    frequency (Lamor frequency) from the cyclotron frequency.
  \item $\omega_a$ depends only on the anomalous magnetic moment $a$
  \item $\omega_a$ depends linearly on the magnetic field $B$
  \end{itemize}
}

%%%%%%%%%%%%%%%%%%%%%%%%%%%%%%%%%%%%%%%%%%%%%%%%%%

\begin{frame}[allowframebreaks]{References}
  \printbibliography
\end{frame}

\end{document}
